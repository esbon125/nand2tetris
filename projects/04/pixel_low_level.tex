\documentclass{article}
\title{Screen indexing - Low Level I/O}
\author{Esteban Bustamante}

\begin{document}
\maketitle
\section*{Considerations:}
\begin{enumerate}
    \item  We have, due to our hardware a max numbers of bits per word.
    \item  Thus, to know how many words will take to have all the rows with our max number of bits per word, we simply:
    $$WpR=\frac{Width\cdot Height}{BpW\cdot Height} $$
\begin{center}
    Where: $BpW=$ Bits per Word and $ WpR=$ Words per row
\end{center}
\item Then, if we exactly want to manipulate one exact bit, we do the following 3 steps:
\begin{itemize}
    \item $word=RAM[offset+WpR*row+col/BpW] $(to obtain the word location)
    \item $ word[col\%BpW]=$ (set the exact bit within the word to 1, with modulo operation)
    \item Commit $word$ to the data memory (RAM).
\end{itemize}
\end{enumerate}
\section*{Hack Programming}
\begin{enumerate}
    \item Registers and memory: add, data and control operations, op codes A:0, C=1;
    \item Branching: conditional statements. Labels between brackets
    \item Variables: declaration with @, no labels.
    \item Iteration: using jump operations.
    \item Pointers: 
    \item Input/Output
\end{enumerate}
\end{document}